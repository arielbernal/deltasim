\documentclass[sigconf]{acmart}

\usepackage{booktabs} % For formal tables
\usepackage{graphicx}  
\usepackage{listings}
%\usepackage{pxfonts}
%\usepackage{color}
\usepackage{bm}
\usepackage{xcolor}
\usepackage[T1]{fontenc}


% Copyright
%\setcopyright{none}
%\setcopyright{acmcopyright}
%\setcopyright{acmlicensed}
%\setcopyright{rightsretained}
%\setcopyright{usgov}
%\setcopyright{usgovmixed}
%\setcopyright{cagov}
%\setcopyright{cagovmixed}

% DOI
%\acmDOI{xx.xxx/xxx_x}

% ISBN
%\acmISBN{xxx-xxxx-xx-xxx/xx/xx}

\copyrightyear{2017} 
\acmYear{2017} 
\setcopyright{rightsretained} 
\acmConference{IWOCL '17}{May 16-18, 2017}{Toronto, Canada}
\acmDOI{http://dx.doi.org/10.1145/3078155.3078183}
\acmISBN{978-1-4503-5214-7/17/05}

\makeatletter
\newcommand\BeraMonottfamily{%
	\def\fvm@Scale{0.75}
	\fontfamily{fvm}\selectfont
}
\makeatother

\definecolor{sh_comment}{rgb}{0.12, 0.38, 0.18 }
\definecolor{sh_keyword}{rgb}{0.37, 0.08, 0.25}
\definecolor{sh_string}{rgb}{0.06, 0.50, 0.68}

\lstset { %
	language={[11]C++},
	%basicstyle=\small,
	%stringstyle=\ttfamily,
	basicstyle= \BeraMonottfamily,
	keywordstyle=\bfseries,
	frame = single,
    stringstyle=\color{sh_string},
    keywordstyle = \color{sh_keyword}\bfseries,
	commentstyle=\color{sh_comment}\itshape,
    lineskip=-0.3em,
	morekeywords={
		vxCreateGenericNode,
		vxSetParameterByIndex,
		vxSetKernelAttribute,
		vxProcessGraph,
		vxAddParameterToKernel,
		clCreateProgramWithSource,
		vxFinalizeKernel,
		clCreateKernel,
		clEnqueueNDRangeKernel,
		clSetKernelArgs,
	},
	escapeinside=!!
}

\definecolor{bostonuniversityred}{rgb}{0.8, 0.0, 0.0}
\definecolor{goldenbrown}{rgb}{0.6, 0.4, 0.08}
\newcommand\setc[1]{\hspace{-1ex}\textcolor{blue}{\bf\BeraMonottfamily{#1}}}
\newcommand\setv[1]{\hspace{-1ex}\textcolor{bostonuniversityred}{\bf\BeraMonottfamily{#1}}}
\newcommand\setd[1]{\hspace{-1ex}\textcolor{goldenbrown}{\bf\BeraMonottfamily{#1}}}


\begin{document}
\title{Inverse and Forward Kinematics for a Delta Printer}
%\titlenote{Produces the permission block, and
%  copyright information}
%\subtitle{Extended Abstract}
%\subtitlenote{The full version of the author's guide is available as
%  \texttt{acmart.pdf} document}

\author{Guadalupe Bernal}
\affiliation{%
  \institution{}
}
\email{guadabernal@gmail.com}

% The default list of authors is too long for headers}
\renewcommand{\shortauthors}{B.Ashbaugh, A.Bernal}


\begin{abstract}
	Inverse and forward kinematics for a delta printer
\end{abstract}

%
% The code below should be generated by the tool at
% http://dl.acm.org/ccs.cfm
% Please copy and paste the code instead of the example below. 
%                                          
\begin{CCSXML}
	<ccs2012>
	<concept>
	<concept_id>10010147.10010169.10010175</concept_id>
	<concept_desc>Computing methodologies~Parallel programming languages</concept_desc>
	<concept_significance>500</concept_significance>
	</concept>
	<concept>
	<concept_id>10010147.10010178.10010224</concept_id>
	<concept_desc>Computing methodologies~Computer vision</concept_desc>
	<concept_significance>500</concept_significance>
	</concept>
	<concept>
	<concept_id>10010147.10010371.10010382.10010383</concept_id>
	<concept_desc>Computing methodologies~Image processing</concept_desc>
	<concept_significance>300</concept_significance>
	</concept>
	</ccs2012>
\end{CCSXML}

\ccsdesc[500]{Computing methodologies~Parallel programming languages}
\ccsdesc[500]{Computing methodologies~Computer vision}
\ccsdesc[300]{Computing methodologies~Image processing}                                         

% We no longer use \terms command
%\terms{Theory}

\keywords{OpenCL, OpenVX, Interoperability, Computer Vision.}

\maketitle

\section{Introduction}



%\begin{figure}[h] 
%	\includegraphics[width=8cm]{OpenVXHeteroGraph.png}
%	\caption {OpenVX heterogeneous graph}
%	\label{OpenVXHeteroGraph}
%\end{figure}

%\begin{figure}[h] 
%	\includegraphics[width=8cm]{OpenVXCLContext.png}
%	\caption {OpenVX - OpenCL Context comparison}
%	\label{OpenVXCLContext}
%\end{figure}

\section{Inverse Kinematics}
The delta printer is modeled by three columns $A, B$ and $C$, each one positioned at a vertex in an equilateral triangle. We define the point $\bm{q} \in $  ${\mathbb{R}^3}$ $\bm{q}=[q_x, q_y, q_z]'$ as the point of extrusion with respect to the bed. 
The effector plane lies above $\bm{q}$ by a distance $h$. 
The distance between the effector plane and the joint at a carriage is defined by $A_c, B_c, C_c$
The height of the carriages with respect to the bed is defined by $A_z, B_z, C_z$.
Then
\begin{equation}
\begin{matrix}
A_z = q_z + A_c + h \\
B_z = q_z + B_c + h \\
C_z = q_z + C_c + h
\end{matrix}
\end{equation}
The length of a the rods is defined by $\ell$ so
The distances between the effector joints and the carriages is defined by $AD, BD, CD$ 
\begin{equation}
\begin{matrix}
\ell^2 = {A_c}^2 + {A_d}^2 \\
\ell^2 = {B_c}^2 + {B_d}^2 \\
\ell^2 = {C_c}^2 + {C_d}^2
\end{matrix}
\end{equation}



\section{Forward Kinematics}
\section{Software Implementation}
\begin{lstlisting}[caption={OpenCL Interop-kernel},label={OpenCLInteropKernel}]
// Get OpenCL context associated with an OpenVX target
cl_context clContext = !\setc{vxGetOpenCLContext}!(vxContext, targt);
// OpenCL standard code for creating kernels
cl_program clProgram = clCreateProgramWithSource(&src, ...);
cl_kernel clKernel0 = clCreateKernel(clProgram, "k0", ...);
cl_kernel clKernel1 = clCreateKernel(clProgram, "k1", ...);
...
// Create an OpenVX kernel for OpenCL interop
vx_kernel vxKernel = !\setc{vxAddOpenCLInteropKernel}!(targets, ...,
!\setv{userFunc}!, !\setv{userVal}!, !\setv{userInit}!, !\setv{userDeinit}!);
// Attach OpenCL kernels to an OpenVX interop kernel
!\setc{vxAddOpenCLKernelToKernel}!(vxKernel, 0, clKernel0);
!\setc{vxAddOpenCLKernelToKernel}!(vxKernel, 1, clKernel1);
// OpenVX standard code for user-kernels
vxAddParameterToKernel(vxKernel, 0, VX_INPUT, ...);
vxAddParameterToKernel(vxKernel, 1, VX_OUTPUT, ...);
vxFinalizeKernel(vxKernel);
vx_node node = vxCreateGenericNode(graph, kernel);
vxSetParameterByIndex(node, 0, inputImage);
vxSetParameterByIndex(node, 1, outputImage);
vxProcessGraph(graph);
\end{lstlisting}
\section{Error Analysis}
\section{Experimentation}
\section{Conclusions}

\let\thefootnote\relax\footnote{Intel and the Intel logo are trademarks of Intel Corporation or its subsidiaries in the U.S. and/or other countries.*Other names and brands may be claimed as the property of others. OpenCL and the OpenCL logo are trademarks of Apple Inc. used by permission by Khronos. OpenVX and the OpenVX logo are trademarks of the Khronos Group Inc.}  
	
\bibliographystyle{ACM-Reference-Format}
\bibliography{sigproc} 

\end{document}
